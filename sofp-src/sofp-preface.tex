
\addchap{Preface}

\global\long\def\gunderline#1{\mathunderline{greenunder}{#1}}%
\global\long\def\bef{\forwardcompose}%
\global\long\def\bbnum#1{\custombb{#1}}%
This book is a reference text and a tutorial that teaches functional
programmers how to reason mathematically about types and code, in
a manner directly relevant to software practice.

The material ranges from introductory to advanced. %
\begin{comment}
Readers will need to learn some difficult concepts through prolonged
mental concentration and effort. 
\end{comment}
The book assumes a certain amount of mathematical experience, at the
level of familiarity with undergraduate algebra or calculus.

The vision of this book is to explain the mathematical theory that
guides the practice of functional programming. So, all mathematical
developments in this book are motivated by practical programming issues
and are accompanied by Scala code illustrating their usage. For instance,
the laws for standard typeclasses (functors, monads, etc.) are first
motivated heuristically through code examples. Only then the laws
are formulated as mathematical equations and proved rigorously. 

To achieve a clearer presentation of the material, the book uses some
non-standard notations (see Appendix~\ref{chap:Appendix-Notations})
and terminology (Appendix~\ref{chap:Appendix-Glossary-of-terms}).
The presentation is self-contained, defining and explaining all required
techniques, notations, and Scala features. Although the code examples
are in Scala, the material in this book also applies to most other
functional programming languages.

All concepts and techniques are motivated and explained as simply
as possible (\textsf{``}but not simpler\textsf{''}) and also clarified via solved
examples and exercises, which should be solved after absorbing the
preceding material. More difficult examples and exercises are marked
by an asterisk ({*}).

A software engineer needs to know only a few fragments of mathematical
theory that will answer questions arising in the practice of functional
programming. So, this book keeps theoretical material at the minimum;
\emph{vita brevis, ars longa}. Chapter~\ref{chap:Applied-functional-type}
discusses the scope of the required theory. The required mathematical
knowledge is limited to first notions of set theory, formal logic,
and category theory. Concepts such as functors or natural transformations
arise organically from the practice of reasoning about code and are
first introduced without reference to category theory. This book does
not use \textsf{``}introduction/elimination rules\textsf{''}, \textsf{``}strong normalization\textsf{''},
\textsf{``}complete partial orders\textsf{''}, \textsf{``}adjoint functors\textsf{''}, \textsf{``}pullbacks\textsf{''},
\textsf{``}co-ends\textsf{''}, or \textsf{``}topoi\textsf{''}, as learning these concepts will neither
help programmers write code nor answer any practical questions about
code. 

This book is also \emph{not} an introduction to current theoretical
research in functional programming. Instead, the focus is on material
known to be practically useful. That includes constructions such as
the \textsf{``}filterable functor\textsf{''} and \textsf{``}applicative contrafunctor\textsf{''} but
excludes a number of theoretical developments that do not (yet?) appear
to have significant applications.

The first part of the book is for beginners in functional programming.
Readers already familiar with functional programming could skim the
glossary (Appendix~\ref{chap:Appendix-Glossary-of-terms}) for unfamiliar
terminology and then start the book with Chapter~\ref{chap:5-Curry-Howard}.

Chapters~\ref{chap:5-Curry-Howard}\textendash \ref{chap:Functors,-contrafunctors,-and}
begin using the code notation, such as Eq.~(\ref{eq:f-functor-exponential-def-of-fmap}).
If that notation still appears hard to follow after going through
Chapters~\ref{chap:5-Curry-Howard}\textendash \ref{chap:Functors,-contrafunctors,-and},
readers will benefit from working through Chapter~\ref{chap:Reasoning-about-code},
which summarizes the code notation more systematically and clarifies
it with additional examples.

All code examples are intended only for explanation and illustration.
As a rule, the code is not optimized for performance or stack safety.

The author thanks Joseph Kim and Jim Kleck for doing some of the exercises
and reporting some errors in earlier versions of this book. The author
also thanks Bill Venners for many helpful comments on the draft, and
Harald Gliebe, Andreas R\"ohler, and Philip Schwarz for contributing
corrections to the text via \texttt{github}. The author is grateful
to Frederick Pitts and several anonymous \texttt{github} contributors
who reported errors in the draft and made helpful suggestions, and
to Barisere Jonathan for valuable assistance with setting up automatic
builds.

\addsec{Formatting conventions used in this book}
\begin{itemize}
\item Text in boldface indicates a new concept or term that is being defined
at that place in the text. Italics means logical emphasis. Example:
\end{itemize}
\begin{quotation}
An \textbf{aggregation\index{aggregation}} is a function from a sequence
of values to a \emph{single} value.
\end{quotation}
\begin{itemize}
\item Equations are numbered per chapter: Eq.~(\ref{eq:prime-formula-function}).
Statements, examples, and exercises are numbered per subsection: Example~\ref{subsec:ch1-aggr-Example-6}
is in subsection~\ref{subsec:Aggregation-solved-examples}, which
belongs to Chapter~\ref{chap:1-Values,-types,-expressions,}.
\item Scala code is written inline using a small monospaced font, such as
\lstinline!flatMap! or \lstinline!val a = "xyz"!. Longer code examples
are written in separate code blocks and may also show the Scala interpreter\textsf{'}s
output for certain lines:
\begin{lstlisting}[mathescape=true]
val s = (1 to 10).toList

scala> s.product
res0: Int = 3628800 
\end{lstlisting}
\item In the introductory chapters, type expressions and code examples are
written in the syntax of Scala. Starting from Chapters~\ref{chap:Higher-order-functions}\textendash \ref{chap:5-Curry-Howard},
the book introduces and uses a new notation for types: e.g., the Scala
type expression \lstinline!((A, B)) => Option[A]! is written as $A\times B\rightarrow\bbnum 1+A$.
Chapters~\ref{chap:Higher-order-functions}\textendash \ref{chap:Reasoning-about-code}
also develop a new notation for code, intended for efficient reasoning
about typeclass laws. For example, the functor composition law is
written in the code notation as:
\[
f^{\uparrow L}\bef g^{\uparrow L}=\left(f\bef g\right)^{\uparrow L}\quad,
\]
where $L$ is a functor and $f^{:A\rightarrow B}$ and $g^{:B\rightarrow C}$
are arbitrary functions of the specified types. The notation $f^{\uparrow L}$
denotes the function $f$ lifted to the functor $L$ and replaces
Scala\textsf{'}s syntax \lstinline!x.map(f)! where \lstinline!x! is of type
\lstinline!L[A]!. The symbol $\bef$ denotes the forward composition
of functions (Scala\textsf{'}s method \lstinline!andThen!). Appendix~\ref{chap:Appendix-Notations}
summarizes these notational conventions for types and code.
\item Frequently used methods of standard typeclasses, such as \lstinline!pure!,
\lstinline!flatMap!, \lstinline!flatten!, \lstinline!filter!, etc.,
are denoted by shorter words and are labeled by the type constructor
they belong to. For instance, the book talks about the methods \lstinline!pure!,
\lstinline!flatten!, and \lstinline!flatMap! for a monad $M$ but
denotes the same methods by $\text{pu}_{M}$, $\text{ftn}_{M}$, and
$\text{flm}_{M}$ when writing code formulas and proofs of laws.
\item Derivations are written in a two-column format. The right column contains
formulas in the code notation. The left column gives an explanation
or indicates the property or law used to derive the expression at
right from the previous expression. A green underline shows the parts
of an expression that will be rewritten in the \emph{next} step:
\begin{align*}
{\color{greenunder}\text{expect to equal }\text{pu}_{M}:}\quad & \gunderline{\text{pu}_{M}^{\uparrow\text{Id}}}\bef\text{pu}_{M}\bef\text{ftn}_{M}\\
{\color{greenunder}\text{lifting to the identity functor}:}\quad & =\text{pu}_{M}\bef\gunderline{\text{pu}_{M}\bef\text{ftn}_{M}}\\
{\color{greenunder}\text{left identity law of }M:}\quad & =\text{pu}_{M}\quad.
\end{align*}
A green underline is sometimes also used at the last step of a derivation,
to indicate the sub-expression that resulted from the most recent
rewriting. Other than providing hints to help  remember the steps,
the green underlines \emph{play no role} in symbolic calculations.
\item The symbol $\square$ is used occasionally to indicate the end of
a definition, a derivation, or a proof.
\end{itemize}

