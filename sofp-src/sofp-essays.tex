
\chapter{\textquotedblleft Applied functional type theory\textquotedblright :
A proposal\label{chap:Applied-functional-type}}

What exactly is the extent of \textsf{``}theory\textsf{''} that a software engineer
should know in order to be a proficient functional programmer? This
book proposes an answer to that question by presenting a coherent
body of theoretical knowledge that, in the author\textsf{'}s view, \emph{is}
the theory that underlies the practice of functional programming and
guides software engineers in writing code. This body of knowledge
may be viewed as a new emerging sub-branch of computer science, tentatively
called \textbf{\index{applied functional type theory}applied functional
type theory} (AFTT). 

In order to discover the proper scope of AFTT, this book appraises
the various inventions made in the field of functional programming
in the last 30 years, such as the \textquotedblleft functional pearls\textquotedblright{}
papers\footnote{\texttt{\href{https://wiki.haskell.org/Research_papers/Functional_pearls}{https://wiki.haskell.org/Research\_papers/Functional\_pearls}}}
and various online tutorials, looking for theoretical material that
has demonstrated its pragmatic usefulness. As a first step towards
formulating AFTT from the ground up, the results are presented in
the form of a tutorial, with motivations and rigorous derivations
of substantially all relevant mathematical facts.

In this book, code examples are written in Scala because the author
is fluent in that language. However, most of the material will work
equally well in Haskell, OCaml, and other FP languages. This is because
AFTT is the science of functional programming and not a set of tricks
specific to Scala or Haskell. An advanced user of any functional programming
language will have to face the same questions and struggle with the
same practical issues.

\section{AFTT is not covered by courses in computer science}

Traditional courses of computer science (algorithms and data structures,
complexity theory, distributed systems, databases, network systems,
compilers, operating systems) are largely not relevant to AFTT. Courses
in programming language theory are more relevant but are not presented
at an appropriate level. To an academic computer scientist, the theory
behind Haskell is \textsf{``}System $F\omega$\textsf{''}, a version of $\lambda$-calculus\index{System Fomega (Haskell)@System $F\omega$ (Haskell)}.\footnote{\texttt{\href{https://babel.ls.fi.upm.es/~pablo/Papers/Notes/f-fw.pdf}{https://babel.ls.fi.upm.es/$\sim$pablo/Papers/Notes/f-fw.pdf}}}
That theory guided the design of the Haskell language and defines
rigorously what a Haskell program means in a mathematical sense. The
theory behind Scala is called the \textsf{``}DOT\textsf{''} (dependent object type)
calculus.\footnote{\texttt{\href{https://www.scala-lang.org/blog/2016/02/03/essence-of-scala.html}{https://www.scala-lang.org/blog/2016/02/03/essence-of-scala.html}}}\index{dependent object type (DOT) calculus}
That theory guided the design of Scala version 3.

However, a practicing Haskell or Scala programmer is not concerned
with designing Haskell or Scala, or with proving theoretical properties
of those languages. Instead, the programmer is mainly concerned with
\emph{using} a chosen programming language to write code. 

Knowing how to prove various properties of System $F\omega$ or DOT
will not actually help programmers to write code. So, these theories
are outside the scope of AFTT. The practice of functional programming
does not require graduate-level theoretical studies.

As an example of theoretical material that \emph{is} within the scope
of AFTT, consider applicative functors (Chapter~\ref{chap:8-Applicative-functors,-contrafunctors}).\index{applicative functors}
It is helpful for a practicing functional programmer to be able to
recognize and use applicative functors. An applicative functor is
a data structure specifying declaratively some operations that can
run independently of each other. Programs may combine these operations,
execute them in parallel, check for validity, or refactor for optimization
or better maintainability.

To use this functionality, the programmer must begin by checking whether
a given data structure satisfies the laws of applicative functors.
In a given application, the choice of a data structure may be dictated
in part by the business logic. The programmer first writes down the
type of that data structure and the code implementing the required
methods. The programmer can then check whether the laws hold. The
data structure and the code may need to be adjusted in order to fit
the definition of an applicative functor and to make the laws hold.

So, the programmer needs to perform a certain amount of symbolic derivations
before coding. The derivations can be done using pen and paper by
writing equations in a concise mathematical notation. Once the laws
are verified, the programmer proceeds to write code.

The mathematical proofs and derivations assure that the chosen data
structure will satisfy the laws of applicative functors, no matter
how the rest of the program is written. So, for example, it is assured
that the relevant operations can be automatically parallelized and
will still work correctly. In this way, AFTT directly guides the programmer
and helps write correct code.

Applicative functors were discovered by practitioners who were using
Haskell to implement parser combinators for compilers. However, applicative
functors are not a feature of Haskell. Rather, they are a design pattern
that may be used in Scala or in any other functional programming language.
A prominent example of an applicative functor is Apache Spark\textsf{'}s \lstinline!RDD!
data type, which is widely used for implementing large-scale parallel
computations.\footnote{\texttt{\href{https://spark.apache.org/docs/latest/rdd-programming-guide.html}{https://spark.apache.org/docs/latest/rdd-programming-guide.html}}}
And yet, no standard computer science course or textbook defines applicative
functors, motivates their laws, explores their structure on examples,
or shows data types that are \emph{not} applicative functors (and
explains why not). 

\section{AFTT is not category theory, type theory, or formal logic}

One often hears that functional programming is based on category theory.\footnote{\texttt{\href{https://www.47deg.com/blog/science-behind-functional-programming/}{https://www.47deg.com/blog/science-behind-functional-programming/}}}
Indeed, the material shown in this book includes a (small) number
of notions from category theory, as well as from formal logic and
type theory. However, software engineers would not benefit from traditional
academic courses in those subjects: their presentation is too abstract
and at the same time lacks specific results necessary for practical
programming. Those courses answer questions that academic mathematicians
have, not questions that practicing functional programmers have.

There exist books intended as presentations of category theory for
computer scientists\footnote{See, e.g., \texttt{\href{https://www.amazon.com/dp/0262660717}{https://www.amazon.com/dp/0262660717}}
or \texttt{\href{https://www.math.mcgill.ca/triples/Barr-Wells-ctcs.pdf}{https://www.math.mcgill.ca/triples/Barr-Wells-ctcs.pdf}}} or for programmers.\footnote{\texttt{\href{https://github.com/hmemcpy/milewski-ctfp-pdf}{https://github.com/hmemcpy/milewski-ctfp-pdf}}}
However, those books do not cover certain concepts relevant to programming,
such as applicative\footnote{Applicative functors are known in mathematics as \textsf{``}monoidal\textsf{''}: \texttt{\href{https://en.wikipedia.org/wiki/Monoidal_functor}{https://en.wikipedia.org/wiki/Monoidal\_functor}}}
or traversable functors. Instead, those books dwell on concepts (e.g.,
limits, enriched categories, topoi) that have no applications in practical
functional programming today.

Typical questions in academic books are \textsf{``}Is $X$ an introduction
rule or an elimination rule\textsf{''} and \textsf{``}Does property $Y$ hold in non-small
categories or only in the category of sets\textsf{''}. Questions a Scala programmer
might ask are \textsf{``}Can we compute a value of type \lstinline!Either[Z, R => A]!
from a value of type \lstinline!R => Either[Z, A]!\textsf{''} and \textsf{``}Is the
type constructor \lstinline!F[A] = Option[(A,A,A)]! a monad or only
an applicative functor\textsf{''}. The scope of AFTT includes answering the
last two questions but \emph{not} the first two.

A software engineer hoping to understand the theory behind functional
programming will not find the concepts of filterable, applicative,
or traversable functors in any currently available books on category
theory, including books intended for programmers. And yet these concepts
are necessary for correct implementations of the important and widely
used operations \lstinline!filter!, \lstinline!zip!, and \lstinline!fold!.

To compensate for the lack of AFTT textbooks, programmers have written
many online tutorials, aiming to explain the theoretical concepts
necessary for practical work. The term \textsf{``}monad tutorial\textsf{''} became
infamous because so many were  posted online.\footnote{\texttt{\href{https://www.johndcook.com/blog/2014/03/03/monads-are-hard-because/}{https://www.johndcook.com/blog/2014/03/03/monads-are-hard-because/}}}
Tutorials were also written about applicative functors, traversable
functors, free monads, etc., showing a real unfulfilled need for presenting
practice-relevant fragments of theory in an applied setting. 

For example, \textsf{``}free monads\textsf{''} became popular in the Scala community
around 2015. Many talks about free monads were presented at Scala
engineering conferences, giving different implementations but never
formulating rigorously the properties required for a piece of code
to be a valid implementation of the free monad. Without knowing the
required mathematical properties of free monads, a programmer cannot
make sure that a given implementation is correct. However, books on
category theory define free monads in a way that is unsuitable for
programming applications (a free monad is an adjoint functor to a
forgetful functor from a Kleisli category to the category of sets).\footnote{\textsf{``}\emph{A monad is just a monoid in the category of endofunctors.
What\textsf{'}s the problem?}\textsf{''} as the well-known joke\index{jokes} goes.\label{fn:A-monad-is-a-monoid-in-category-of-endofunctors-big-deal}
For background information about that joke, see \texttt{\href{https://stackoverflow.com/questions/3870088/}{https://stackoverflow.com/questions/3870088/}};
a related joke is in footnote~\ref{fn:Whats-the-big-deal-monad-transformers}
on page~\pageref{fn:Whats-the-big-deal-monad-transformers}.} Such \textsf{``}academic\textsf{''} definitions can be used neither as guidance for
writing code or checking code correctness, nor as a conceptual explanation
that a learner would find helpful.

Perhaps the best selection of AFTT tutorial material today can be
found in the \textsf{``}Haskell Wikibooks\textsf{''}.\footnote{\texttt{\href{https://en.wikibooks.org/wiki/Haskell}{https://en.wikibooks.org/wiki/Haskell}}}
However, those tutorials are incomplete and limited to explaining
the use of Haskell. Many of them are suitable neither as a first introduction
nor as a reference on AFTT. Also, the Haskell Wikibooks tutorials
rarely show any derivations of laws or explain the required techniques.

Apart from referring to some notions from category theory, AFTT also
uses concepts from type theory and formal logic. However, existing
textbooks on type theory and formal logic focus on domain theory and
proof theory. From a practicing programmer\textsf{'}s viewpoint, these books
present a lot of difficult-to-learn material that will not help in
writing code. At the same time, those academic books never mention
practical techniques used in many functional programming libraries
today, such as reasoning about and implementing types with quantifiers,
types parameterized by type constructors, or partial type-level functions
(known as \textsf{``}typeclasses\textsf{''}).

The proper scope of AFTT is to help the programmer with practical
tasks such as:
\begin{enumerate}
\item Deciding whether two data types are equivalent and implementing the
isomorphism transformations. For example, the Scala type \lstinline!(A, Either[B, C])!
is equivalent to \lstinline!Either[(A, B), (A, C)]!, but the type
\lstinline!A => Either[B, C]! is \emph{not} equivalent to \lstinline!Either[A => B, A => C]!.
\item Checking whether a definition of a recursive type is \textsf{``}valid\textsf{''},
i.e., does not lead to infinite loops. A simple example of an \textsf{``}invalid\textsf{''}
recursive type definition in Scala is \lstinline!class Bad(x: Bad)!.
A small change transforms that example into a \textsf{``}valid\textsf{''} recursive
type: \lstinline!class Good(x: Option[Good])!.
\item Deciding whether a function with a given type signature can be implemented.
For example, 
\begin{lstlisting}
def f[Z,A,R]: (R => Either[Z, A]) => Either[Z, R => A] = ???   // Cannot be implemented.
def g[Z,A,R]: Either[Z, R => A] => (R => Either[Z, A]) = ???   // Can be implemented.
\end{lstlisting}
\item Deriving an implementation of a function from its type signature and
checking required laws. For example, deriving the \lstinline!flatMap!
method and checking its laws for the \lstinline!Reader! monad:
\begin{lstlisting}
def flatMap[Z, A, B](r: Z => A)(f: A => Z => B): Z => B = ???
\end{lstlisting}
\item Deriving a simpler but equivalent code by calculating with functions
and laws.
\end{enumerate}
These are real-world applications of type theory and the Curry-Howard
correspondence, but existing books on type theory and logic do not
give practical recipes for performing these tasks.\footnote{Task 5 is addressed in several programming-oriented books such as
\emph{Pearls of functional algorithm design} by \index{Richard Bird}Richard
Bird (\texttt{\href{https://www.cambridge.org/9780521513388}{https://www.cambridge.org/9780521513388}}).}

Books such as \emph{Scala with Cats},\footnote{\texttt{\href{https://underscore.io/books/scala-with-cats/}{https://underscore.io/books/scala-with-cats/}}}
\emph{Functional programming simplified},\footnote{\texttt{\href{https://alvinalexander.com/scala/functional-programming-simplified-book}{https://alvinalexander.com/scala/functional-programming-simplified-book}}}
and \emph{Functional programming for mortals}\footnote{\texttt{\href{http://www.lulu.com/shop/search.ep?contributorId=1600066}{http://www.lulu.com/shop/search.ep?contributorId=1600066}}}
are primarily focused on explaining practical aspects of functional
programming and do not derive the mathematical laws for, e.g., filterable,
monadic, applicative, or traversable functors.

The only currently available Scala-based AFTT textbook is \emph{Functional
Programming in Scala}.\footnote{\texttt{\href{https://www.manning.com/books/functional-programming-in-scala}{https://www.manning.com/books/functional-programming-in-scala}}}
It balances practical coding with theoretical developments and laws.
\emph{Program design by calculation}\footnote{\texttt{\href{http://www4.di.uminho.pt/~jno/ps/pdbc.pdf}{http://www4.di.uminho.pt/$\sim$jno/ps/pdbc.pdf}}}
is another (Haskell-oriented) AFTT book in progress. The present book
is written at about the same level but aims at better motivation for
mathematical concepts and a wider range of pedagogical examples that
help build the necessary intuition and facility with the techniques
of formal derivation. 

Figures~\ref{fig:Randomly-chosen-pages-1}\textendash \ref{fig:Randomly-chosen-pages}
illustrate the difference between AFTT books, programming books, and
academic science books, by showing randomly chosen pages from such
books. One gets a visual impression that programming-oriented books
contain code examples and explanations in words but no formal derivations.
Books on AFTT, as well as books on mathematics and science, will typically
show equations, diagrams, and derivations. The present book contains
both code examples and mathematical manipulations.

\begin{figure}
\begin{centering}
\includegraphics[height=2.51cm]{random-pages/random-pages-from-fpsimplified-pdf-00}\includegraphics[height=2.51cm]{random-pages/random-pages-from-fpsimplified-pdf-01}\includegraphics[height=2.51cm]{random-pages/random-pages-from-fpsimplified-pdf-02}\includegraphics[height=2.51cm]{random-pages/random-pages-from-fpsimplified-pdf-03}\includegraphics[height=2.51cm]{random-pages/random-pages-from-fpsimplified-pdf-04}\includegraphics[height=2.51cm]{random-pages/random-pages-from-fpsimplified-pdf-05}\includegraphics[height=2.51cm]{random-pages/random-pages-from-fpsimplified-pdf-06}\includegraphics[height=2.51cm]{random-pages/random-pages-from-fpsimplified-pdf-07}
\par\end{centering}
\begin{centering}
\vspace{-0.3\baselineskip}
\par\end{centering}
\begin{centering}
\emph{Functional programming simplified}, by A.~Alexander
\par\end{centering}
\begin{centering}
\vspace{1\baselineskip}
\par\end{centering}
\begin{centering}
\includegraphics[height=2.51cm]{random-pages/random-pages-from-fpmortals-pdf-00}\includegraphics[height=2.51cm]{random-pages/random-pages-from-fpmortals-pdf-01}\includegraphics[height=2.51cm]{random-pages/random-pages-from-fpmortals-pdf-02}\includegraphics[height=2.51cm]{random-pages/random-pages-from-fpmortals-pdf-03}\includegraphics[height=2.51cm]{random-pages/random-pages-from-fpmortals-pdf-04}\includegraphics[height=2.51cm]{random-pages/random-pages-from-fpmortals-pdf-06}\includegraphics[height=2.51cm]{random-pages/random-pages-from-fpmortals-pdf-07}\includegraphics[height=2.51cm]{random-pages/random-pages-from-fpmortals-pdf-08}
\par\end{centering}
\begin{centering}
\emph{\vspace{-1.6\baselineskip}
}
\par\end{centering}
\begin{centering}
\emph{Functional programming for mortals}, by S.~Halliday
\par\end{centering}
\begin{centering}
\vspace{1\baselineskip}
\par\end{centering}
\begin{centering}
\includegraphics[height=2.51cm]{random-pages/random-pages-from-volpe-pdf-01}\includegraphics[height=2.51cm]{random-pages/random-pages-from-volpe-pdf-03}\includegraphics[height=2.51cm]{random-pages/random-pages-from-volpe-pdf-04}\includegraphics[height=2.51cm]{random-pages/random-pages-from-volpe-pdf-05}\includegraphics[height=2.51cm]{random-pages/random-pages-from-volpe-pdf-06}\includegraphics[height=2.51cm]{random-pages/random-pages-from-volpe-pdf-07}\includegraphics[height=2.51cm]{random-pages/random-pages-from-volpe-pdf-08}\includegraphics[height=2.51cm]{random-pages/random-pages-from-volpe-pdf-09}
\par\end{centering}
\begin{centering}
\emph{\vspace{-1.9\baselineskip}
}
\par\end{centering}
\begin{centering}
\emph{Practical functional programming in Scala}, by G.~Volpe (\texttt{\small{}\href{https://leanpub.com/pfp-scala}{https://leanpub.com/pfp-scala}})
\par\end{centering}
\begin{centering}
\vspace{1\baselineskip}
\par\end{centering}
\begin{centering}
\includegraphics[height=2.51cm]{random-pages/random-pages-from-kalinin-pdf-01}\includegraphics[height=2.51cm]{random-pages/random-pages-from-kalinin-pdf-02}\includegraphics[height=2.51cm]{random-pages/random-pages-from-kalinin-pdf-03}\includegraphics[height=2.51cm]{random-pages/random-pages-from-kalinin-pdf-04}\includegraphics[height=2.51cm]{random-pages/random-pages-from-kalinin-pdf-06}\includegraphics[height=2.51cm]{random-pages/random-pages-from-kalinin-pdf-07}\includegraphics[height=2.51cm]{random-pages/random-pages-from-kalinin-pdf-08}\includegraphics[height=2.51cm]{random-pages/random-pages-from-kalinin-pdf-09}
\par\end{centering}
\vspace{-0.6\baselineskip}

\begin{centering}
\emph{Mastering advanced Scala}, by D.~Kalinin (\texttt{\small{}\href{https://leanpub.com/mastering-advanced-scala}{https://leanpub.com/mastering-advanced-scala}})
\par\end{centering}
\caption{Randomly chosen pages from books on Scala programming.\label{fig:Randomly-chosen-pages-1}}
\end{figure}

\begin{figure}
\begin{centering}
\includegraphics[height=2.51cm]{random-pages/random-pages-from-hefferon-pdf-00}\includegraphics[height=2.51cm]{random-pages/random-pages-from-hefferon-pdf-01}\includegraphics[height=2.51cm]{random-pages/random-pages-from-hefferon-pdf-03}\includegraphics[height=2.51cm]{random-pages/random-pages-from-hefferon-pdf-04}\includegraphics[height=2.51cm]{random-pages/random-pages-from-hefferon-pdf-05}\includegraphics[height=2.51cm]{random-pages/random-pages-from-hefferon-pdf-06}\includegraphics[height=2.51cm]{random-pages/random-pages-from-hefferon-pdf-07}
\par\end{centering}
\vspace{-0.4\baselineskip}

\begin{centering}
\emph{Linear algebra}, by J.~Hefferon (\texttt{\small{}\href{http://joshua.smcvt.edu/linearalgebra/}{http://joshua.smcvt.edu/linearalgebra/}})
\par\end{centering}
\begin{centering}
\vspace{0.6\baselineskip}
\par\end{centering}
\begin{centering}
\includegraphics[height=2.51cm]{random-pages/random-pages-from-kibble-pdf-00}\includegraphics[height=2.51cm]{random-pages/random-pages-from-kibble-pdf-01}\includegraphics[height=2.51cm]{random-pages/random-pages-from-kibble-pdf-02}\includegraphics[height=2.51cm]{random-pages/random-pages-from-kibble-pdf-03}\includegraphics[height=2.51cm]{random-pages/random-pages-from-kibble-pdf-04}\includegraphics[height=2.51cm]{random-pages/random-pages-from-kibble-pdf-05}\includegraphics[height=2.51cm]{random-pages/random-pages-from-kibble-pdf-06}\includegraphics[height=2.51cm]{random-pages/random-pages-from-kibble-pdf-09}
\par\end{centering}
\vspace{-0.6\baselineskip}

\begin{centering}
\emph{Classical mechanics}, by T.~W.~B.~Kibble and F.~H.~Berkshire
(\texttt{\small{}\href{https://archive.org/details/116772449ClassicalMechanics}{https://archive.org/details/116772449ClassicalMechanics}})
\par\end{centering}
\vspace{0.6\baselineskip}

\caption{Randomly chosen pages from books on mathematics and physics.\label{fig:Randomly-chosen-pages-2}}
\end{figure}

\begin{figure}
\begin{centering}
\includegraphics[height=2.51cm]{random-pages/random-pages-from-fpis-pdf-00}\includegraphics[height=2.51cm]{random-pages/random-pages-from-fpis-pdf-02}\includegraphics[height=2.51cm]{random-pages/random-pages-from-fpis-pdf-04}\includegraphics[height=2.51cm]{random-pages/random-pages-from-fpis-pdf-05}\includegraphics[height=2.51cm]{random-pages/random-pages-from-fpis-pdf-06}\includegraphics[height=2.51cm]{random-pages/random-pages-from-fpis-pdf-07}\includegraphics[height=2.51cm]{random-pages/random-pages-from-fpis-pdf-08}
\par\end{centering}
\vspace{-0\baselineskip}

\begin{centering}
\emph{Functional programming in Scala}, by P.~Chiusano and R.~Bjarnason
\par\end{centering}
\begin{centering}
\vspace{1\baselineskip}
\par\end{centering}
\begin{centering}
\includegraphics[height=2.51cm]{random-pages/random-pages-from-pdbc-pdf-00}\includegraphics[height=2.51cm]{random-pages/random-pages-from-pdbc-pdf-01}\includegraphics[height=2.51cm]{random-pages/random-pages-from-pdbc-pdf-02}\includegraphics[height=2.51cm]{random-pages/random-pages-from-pdbc-pdf-03}\includegraphics[height=2.51cm]{random-pages/random-pages-from-pdbc-pdf-04}\includegraphics[height=2.51cm]{random-pages/random-pages-from-pdbc-pdf-05}\includegraphics[height=2.51cm]{random-pages/random-pages-from-pdbc-pdf-06}\includegraphics[height=2.51cm]{random-pages/random-pages-from-pdbc-pdf-07}
\par\end{centering}
\vspace{-0.7\baselineskip}

\begin{centering}
\emph{Program design by calculation}, by J.~N.~Oliveira
\par\end{centering}
\begin{centering}
\vspace{1\baselineskip}
\par\end{centering}
\begin{centering}
\includegraphics[height=2.51cm]{random-pages/random-pages-from-sofp-pdf-00}\includegraphics[height=2.51cm]{random-pages/random-pages-from-sofp-pdf-01}\includegraphics[height=2.51cm]{random-pages/random-pages-from-sofp-pdf-02}\includegraphics[height=2.51cm]{random-pages/random-pages-from-sofp-pdf-05}\includegraphics[height=2.51cm]{random-pages/random-pages-from-sofp-pdf-06}\includegraphics[height=2.51cm]{random-pages/random-pages-from-sofp-pdf-07}\includegraphics[height=2.51cm]{random-pages/random-pages-from-sofp-pdf-08}
\par\end{centering}
\vspace{-0.3\baselineskip}

\begin{centering}
\emph{The science of functional programming} (this book)
\par\end{centering}
\vspace{1\baselineskip}

\caption{Randomly chosen pages from books on applied functional type theory.\label{fig:Randomly-chosen-pages}}
\end{figure}


\chapter{Essay: Software engineers and software artisans}

Let us examine what we ordinarily understand by \emph{engineering}
as opposed to artisanship or craftsmanship. It will then become apparent
that today\textsf{'}s computer programmers must be viewed as \textsf{``}software artisans\textsf{''}
rather than software engineers.\footnote{The book reviewed in \texttt{\href{https://www.developerdotstar.com/mag/bookreviews/bitner_craftsmanship.html}{https://www.developerdotstar.com/mag/bookreviews/bitner\_craftsmanship.html}}
proposes a different definition of \textsf{``}software artisans\textsf{''} than this
book, but agrees that software developers work largely as artisans.}

\section{Engineering disciplines }

Consider the way mechanical engineers, chemical engineers, or electrical
engineers work, and look at the studies they require for becoming
proficient in their fields.

A mechanical engineer studies calculus, linear algebra, differential
geometry, and several areas of physics such as theoretical mechanics,
thermodynamics, and elasticity theory,\footnote{\texttt{\href{https://www.colorado.edu/mechanical/academics/undergraduate-program/curriculum}{https://www.colorado.edu/mechanical/academics/undergraduate-program/curriculum}}}
and then uses calculations to guide the design of a bridge. A chemical
engineer studies chemistry, thermodynamics, calculus, linear algebra,
differential equations, some areas of physics such as thermodynamics
and kinetic theory,\footnote{\texttt{\href{https://www.colorado.edu/engineering/sample-undergraduate-curriculum-chemical}{https://www.colorado.edu/engineering/sample-undergraduate-curriculum-chemical}}}
and uses calculations to guide the design of a chemical process. An
electrical engineer studies advanced calculus, linear algebra, and
several areas of physics such as electrodynamics and quantum theory,\footnote{\texttt{\href{http://archive.is/XYLyE}{http://archive.is/XYLyE}}}
and uses calculations to design an antenna or a microchip.

The common pattern is that engineers use mathematics and natural sciences
in order to create new devices. Mathematical calculations and scientific
reasoning are performed \emph{before} designing or building a real
machine.

Some of the studies required for engineers include arcane abstract
concepts such as the \textsf{``}Fourier transform of the delta function\textsf{''}\footnote{\texttt{\href{https://www.youtube.com/watch?v=KAbqISZ6SHQ}{https://www.youtube.com/watch?v=KAbqISZ6SHQ}}}
and the \textsf{``}inverse $Z$-transform\textsf{''}\footnote{\texttt{\href{http://archive.is/SsJqP}{http://archive.is/SsJqP}}}
in digital signal processing, \textsf{``}rank 4 tensors\textsf{''}\footnote{\texttt{\href{https://serc.carleton.edu/NAGTWorkshops/mineralogy/mineral_physics/tensors.html}{https://serc.carleton.edu/NAGTWorkshops/mineralogy/mineral\_physics/tensors.html}}}
in calculations of elasticity of materials, \textsf{``}Lagrangians with non-holonomic
constraints\textsf{''}\footnote{\texttt{\href{https://arxiv.org/abs/math/0008147}{https://arxiv.org/abs/math/0008147}}}
in robotics, and the \textsf{``}Gibbs free energy\textsf{''} in chemical reactor design.\footnote{\texttt{\href{https://www.amazon.com/Introduction-Chemical-Engineering-Kinetics-Reactor/dp/1118368258}{https://www.amazon.com/Introduction-Chemical-Engineering-Kinetics-Reactor/dp/1118368258}}}

To be sure, a significant part of what engineers do is not covered
by any theory: the \emph{know-how}, the informal reasoning, the traditional
knowledge passed on from expert to novice,  \textemdash{} all those
skills that are hard to formalize are important. Nevertheless, engineering
is crucially based on natural science and mathematics for some of
its decision-making about new designs.

\section{Artisanship: Trades and crafts }

Now consider what kinds of things shoemakers, plumbers, or home painters
do, and what they have to learn in order to become proficient in their
profession.

A novice shoemaker, for example, begins by copying some drawings\footnote{\texttt{\href{https://youtu.be/cY5MY0czMAk?t=141}{https://youtu.be/cY5MY0czMAk?t=141}}}
and goes on to cutting leather in a home workshop. Apprenticeships
proceed via learning by doing, with comments and instruction from
an expert. After a few years of study (for example, a painter apprenticeship
in California can be as short as 2 years\footnote{\texttt{\href{http://www.calapprenticeship.org/programs/painter_apprenticeship.php}{http://www.calapprenticeship.org/programs/painter\_apprenticeship.php}}}),
a new artisan is ready to start productive work. 

All trades operate entirely from tradition and practical experience.
It takes a prolonged learning effort to become a good artisan in any
profession. But the trades do not require academic study because there
is no formal theory from which to proceed. There are no Fourier transforms
applied to delta functions, no Lagrangians with non-holonomic constraints,
no fourth rank tensors to calculate, nor any differential equations
to solve.

Artisans do not study science or mathematics because their professions
do not make use of any formal theory for guiding their designs or
processes.

\section{Programmers today are artisans, not engineers }

Programmers are \emph{not engineers} in the sense normally used in
the engineering professions.

\subsection{No requirements of licensing or formal study}

Mechanical, electrical, chemical engineers are required to pass a
license exam to become certified to work. The license exam certifies
that the person is proficient in applying a known set of engineering
tools and methods. But in software engineering, no certifications
or licenses are required for the job (although many certification
programs exist).

Working software engineers are also not required to have studied software
engineering or computer science (CS). According to a recent Stack
Overflow survey,\footnote{\texttt{\href{https://thenextweb.com/insider/2016/04/23/dont-need-go-college-anymore-programmer/}{https://thenextweb.com/insider/2016/04/23/dont-need-go-college-anymore-programmer/}}}
about 56\% of working programmers have no CS degree. The author of
this book is a self-taught programmer who has degrees in physics but
never formally studied CS or taken any academic courses in algorithms,
data structures, computer networks, compilers, programming languages,
or other standard CS topics. 

A large fraction of successful programmers have no college degrees
and perhaps \emph{never} studied formally. They acquired all their
knowledge and skills through self-study and practical work. A prominent
example is Robert C.~Martin\index{Robert C.~Martin},\footnote{\texttt{\href{https://en.wikipedia.org/wiki/Robert_C._Martin}{https://en.wikipedia.org/wiki/Robert\_C.\_Martin}}}
an outspoken guru in the arts of programming. He routinely refers
to programmers as artisans\footnote{\texttt{\href{https://blog.cleancoder.com/uncle-bob/2013/02/01/The-Humble-Craftsman.html}{https://blog.cleancoder.com/uncle-bob/2013/02/01/The-Humble-Craftsman.html}}}
and uses the appropriate imagery: novices and masters, trade and craft,
the honor of the guild, etc. He compares programmers to plumbers,
electricians, lawyers, and surgeons, but never to mathematicians,
physicists, or engineers of any kind. According to one of his blog
posts,\footnote{\texttt{\href{https://blog.cleancoder.com/uncle-bob/2013/11/25/Novices-Coda.html}{https://blog.cleancoder.com/uncle-bob/2013/11/25/Novices-Coda.html}}}
he started working at age 17 as a self-taught programmer. He never
went to college and holds no degrees.\footnote{\texttt{\href{https://hashnode.com/post/i-am-robert-c-martin-uncle-bob-ask-me-anything-cjr7pnh8g000k2cs18o5nhulp/2}{https://hashnode.com/post/i-am-robert-c-martin-uncle-bob-ask-me-anything-cjr7pnh8g000k2cs18o5nhulp/2}}}
It is clear that R.~C.~Martin \emph{is} an expert craftsman and
that he did \emph{not} need academic study to master the craft of
programming.

Here is another opinion\footnote{\texttt{\href{http://archive.is/tAKQ3}{http://archive.is/tAKQ3}}}
(emphasis is theirs):
\begin{quotation}
{\small{}Software Engineering is unique among the STEM careers in
that it absolutely does }\emph{\small{}not}{\small{} require a college
degree to be successful. It most certainly does not require licensing
or certification. }\emph{\small{}It requires experience}{\small{}.}{\small\par}
\end{quotation}
This description fits a career in crafts \textemdash{} but certainly
not a career, say, in electrical engineering.

The high demand for software developers gave rise to \textsf{``}developer
boot camps\textsf{''}\footnote{\texttt{\href{http://archive.is/GkOL9}{http://archive.is/GkOL9}}}
\textemdash{} vocational schools that educate new programmers in a
few months through purely practical training, with no formal theory
or mathematics involved. These vocational schools are successful\footnote{\texttt{\href{http://archive.is/E9FXP}{http://archive.is/E9FXP}}}
in job placement. But it is unimaginable that a $6$-month crash course
or even a $2$-year vocational school could prepare engineers to work
successfully on designing, say, analog quantum computers\footnote{\texttt{\href{https://www.dwavesys.com/quantum-computing}{https://www.dwavesys.com/quantum-computing}}}
without ever learning quantum physics or calculus.

\subsection{No mathematical formalism guides software development}

Most books on software engineering contain no formulas or equations,
no mathematical derivations, and no precise definitions of the various
technical terms they are using (such as \textsf{``}object-oriented\textsf{''} or \textsf{``}module\textsf{'}s
responsibilities\textsf{''}). Some of those books\footnote{E.g., \texttt{\href{https://www.amazon.com/Object-Oriented-Software-Engineering-Unified-Methodology/dp/0073376256}{https://www.amazon.com/Object-Oriented-Software-Engineering-Unified-Methodology/dp/0073376256}}}
also have almost no program code in them. Some of those books are
written by practitioners such as R.\ C.\ Martin never studied any
formalisms and do not think in terms of formalisms. Instead, they
summarize their programming experience in vaguely formulated heuristic
\textquotedblleft principles\textquotedblright .\footnote{\texttt{\href{https://blog.cleancoder.com/uncle-bob/2016/03/19/GivingUpOnTDD.html}{https://blog.cleancoder.com/uncle-bob/2016/03/19/GivingUpOnTDD.html}}}
The programmers are told: \textsf{``}code is about detail\textsf{''}, \textsf{``}never abandon
the big picture\textsf{''}, \textsf{``}avoid tight coupling in your modules\textsf{''}, \textsf{``}a
class must serve a single responsibility\textsf{''}, \textsf{``}strive for good interfaces\textsf{''},
etc. 

In contrast, textbooks on mechanical or electrical engineering include
a significant amount of mathematics. The design of a microwave antenna
is guided not by an \textsf{``}open and closed module principle\textsf{''} but by
solving the relevant differential equations\footnote{\texttt{\href{https://youtu.be/8KpfVsJ5Jw4?t=447}{https://youtu.be/8KpfVsJ5Jw4?t=447}}}
of electrodynamics.

Another example of programmers\textsf{'} avoidance of mathematical tools is
given by the \textsf{``}Liskov substitution principle\index{Liskov substitution principle}\textsf{''}
for subtyping\index{subtyping}.\footnote{See \texttt{\href{https://en.wikipedia.org/wiki/Liskov_substitution_principle}{https://en.wikipedia.org/wiki/Liskov\_substitution\_principle}}.
The LSP always holds in functional programming if values are immutable
and subtyping is viewed as an automatic type conversion function (see
Definition~\ref{subsec:Definition-subtyping}). A property $\phi(y)$
is rewritten as $\phi(c(y))$ by inserting a suitable type conversion
function, $c:S\rightarrow T$. Since $c(y)$ has type $T$ and $\phi(x)$
holds for all values $x:T$, we find that the property $\phi(c(y))$
holds automatically.} Its rigorous formulation (\textsf{``}for any property $\phi(x)$ that holds
for all $x$ of type $T$, and for any subtype $S$ of $T$, the property
$\phi(y)$ must also hold for all $y$ of type $S$\textsf{''}) is not used
by programmers. Instead, the literature on object-oriented programming
formulates the principle as \textsf{``}objects of type $T$ may be substituted
by objects of type $S$ while keeping the correctness of the program\textsf{''}.
This formulation \index{object-oriented programming} is both vague
(it does not specify how to choose the substituted objects of type
$S$) and, strictly speaking, incorrect: If the program contains a
function $f(t)$ where $t$ is a value of type $T$, it is not always
possible to find some value $s$ of type $S$ such that $f(s)=f(t)$.
The reason is that some subtyping relations are not surjective, as
shown in Section~\ref{subsec:Subtyping-with-injective} of this book.

Donald Knuth\textsf{'}s classic textbook \textsf{``}\emph{The Art of Programming}\textsf{''}
indeed treats programming as an art and not as a science. Knuth shows
many algorithms and derives their mathematical properties but gives
almost no examples of realistic program code and does not provide
any theory that could guide programmers in actually \emph{writing}
programs (say, choosing the data types to be used). Knuth assumes
that the reader who understands the mathematical properties of an
algorithm will be able \emph{somehow} to write correct code.

The books \textsf{``}The Science of Programming\textsf{''}\footnote{\texttt{\href{https://www.amazon.com/Science-Programming-Monographs-Computer/dp/0387964800}{https://www.amazon.com/Science-Programming-Monographs-Computer/dp/0387964800}}}
and \textsf{``}Program derivation\textsf{''}\footnote{\texttt{\href{https://www.amazon.com/Program-Derivation-Development-Specifications-International/dp/0201416247}{https://www.amazon.com/Program-Derivation-Development-Specifications-International/dp/0201416247}}}
are attempts to provide a mathematical basis for writing programs
starting from formal specifications. The books give some methods that
guide programmers in writing code and at the same time produce a proof
that the code conforms to the specification. However, the scope of
proposed methods is limited to designing algorithms for iterative
manipulation of data, such as sorting and searching algorithms. The
procedures suggested in those books are far from a formal mathematical
\emph{derivation} of a wide range of software programs from specifications.
In any case, most programmers today are unaware of these books and
do not use the methods explained there, even when those methods could
apply.

Today\textsf{'}s computer science courses do not teach a true engineering approach
to software construction. Some courses teach analysis of programs
using mathematical methods. Two such methods are complexity analysis\footnote{\texttt{\href{https://www.cs.cmu.edu/~adamchik/15-121/lectures/Algorithmic\%20Complexity/complexity.html}{https://www.cs.cmu.edu/$\sim$adamchik/15-121/lectures/Algorithmic\%20Complexity/complexity.html}}}
(the \textsf{``}big-$O$ notation\textsf{''}) and formal verification.\footnote{\texttt{\href{https://en.wikipedia.org/wiki/Formal_verification}{https://en.wikipedia.org/wiki/Formal\_verification}}}
But programs are analyzed or verified only \emph{after} they are somehow
written. Theory does not guide the actual \emph{process} of writing
code: it does not define good ways of organizing the code (e.g., how
to decompose the code into modules, classes, or functions) and does
not tell programmers which data structures and type signatures of
functions will be useful to implement. Programmers make these design
decisions on the basis of experience and intuition, trial-and-error,
copy-paste, guesswork, and debugging. 

In a sense, program analysis and verification is analogous to writing
mathematical equations for the surface of a shoe made by a fashion
designer. The resulting \textsf{``}shoe equations\textsf{''} are mathematically rigorous
and can be analyzed or \textsf{``}verified\textsf{''}. But the equations are merely
written after the fact, they do not guide the fashion designers in
actually making shoes. It is understandable that fashion designers
do not study the mathematical theory of surfaces.

\subsection{Programmers avoid academic terminology }

Programmers jokingly grumble about terms such as \textsf{``}functor\textsf{''}, \textsf{``}monad\textsf{''},
or \textsf{``}lambda-functions\textsf{''}:
\begin{quote}
{\small{}Those fancy words used by functional programmers purists
really annoy me. Monads, functors... Nonsense!!! }\footnote{\texttt{\href{http://archive.is/65K3D}{http://archive.is/65K3D}}}
\end{quote}
Perhaps only a small minority of software developers complain about
this, as the majority seems to be unaware of \textsf{``}applicative functors\textsf{''},
\textsf{``}free monads\textsf{''}, and other arcane terminology. Indeed, that sort
of terminology is intentionally avoided by most books and tutorials
aimed at programmers.

But why would a software \emph{engineer} wince at \textsf{``}functors\textsf{''} or
at having to verify the laws of a \textsf{``}monad\textsf{''}? Other branches of engineering
use lots of terminology that is far from self-explanatory and requires
some study. Chemical engineers learn about \textsf{``}Gibbs free energy\textsf{''},
which is a technical term that denotes a certain function. (It does
not mean getting energy from J.~W.~Gibbs\index{Josiah Willard Gibbs}
for free.) Chemical engineers accept the need for studying \textsf{``}phase
diagrams\textsf{''} or \textsf{``}Fourier\textsf{'}s law\textsf{''}. Electrical engineers do not avoid
\textsf{``}Fourier transforms\textsf{''} or \textsf{``}delta functions\textsf{''} because those are
weird things to say. Mechanical engineers take it for granted that
they need \textsf{``}rank 4 tensors\textsf{''}, \textsf{``}Lagrangians\textsf{''}, and \textsf{``}non-holonomic
constraints\textsf{''}. The arcane terminology seems to be the least of their
difficulties, as their textbooks are full of complicated equations
and long derivations.

Textbooks on true software engineering would have been full of equations
and derivations, teaching engineers how to perform certain calculations
that are required \emph{before} starting to write code.

\section{Towards true engineering in software}

It is now clear that we do not presently have true software engineering.
The people employed under that job title are actually artisans. They
work using artisanal methods, and their culture and processes are
that of a crafts guild.

True software engineering means having a mathematical theory that
guides the process of writing programs, \textemdash{} not just theory
that describes or analyzes programs after they are \emph{somehow}
written.

It is not enough that the numerical methods required for physics or
the matrix calculations required for data science are \textsf{``}mathematical\textsf{''}.
These programming tasks are indeed formulated using mathematical theory.
However, mathematical \emph{subject matter} (aerospace control, physics
or astronomy simulations, or statistics) does not mean that mathematics
is used to guide the process of writing code. Data scientists, aerospace
engineers, and physicists almost always work as artisans when converting
their computations into program code.

We expect that software engineers\textsf{'} textbooks should be full
of equations and derivations. What theory would those equations represent?

This theory is what this book calls \textbf{applied functional type
theory}\index{applied functional type theory} (see Chapter~\ref{chap:Applied-functional-type}).
It represents the mathematical foundation of the modern practice of
functional programming, as implemented in languages such as OCaml,
Haskell, and Scala. This theory is a blend of set theory, category
theory, and logical proof theory, adapted for the needs of programmers.
It has been in development since late 1990s and is still being actively
worked on by a community of software practitioners and academic computer
scientists.

To appreciate that functional programming, unlike any other programming
paradigm, is based on a \emph{theory that guides coding}, we can look
at some recent software engineering conferences such as \textsf{``}Scala By
the Bay\textsf{''}\footnote{\texttt{\href{http://2015.scala.bythebay.io/}{http://2015.scala.bythebay.io/}}}
or BayHac,\footnote{\texttt{\href{http://bayhac.org/}{http://bayhac.org/}}}
or at the numerous FP-related online tutorials and blogs. We cannot
fail to notice that speakers devote significant time to a peculiar
kind of applied mathematical reasoning. Rather than focusing on one
or another API or algorithm, as it is often the case with other software
engineering blogs or presentations, an FP speaker describes a \emph{mathematical
structure} \textemdash{} such as the \textsf{``}applicative functor\textsf{''}\footnote{\texttt{\href{http://www.youtube.com/watch?v=bmIxIslimVY}{http://www.youtube.com/watch?v=bmIxIslimVY}}}
or the \textsf{``}free monad\textsf{''}\footnote{\texttt{\href{http://www.youtube.com/watch?v=U0lK0hnbc4U}{http://www.youtube.com/watch?v=U0lK0hnbc4U}}}
\textemdash{} and illustrates its use for practical coding.

These people are not graduate students showing off their theoretical
research. They are practitioners, software engineers who use FP on
their jobs. It is just the nature of FP that certain mathematical
tools and constructions are directly applicable to practical programming
tasks.

These mathematical tools are not mere tricks for a specific programming
language; they apply equally to all FP languages. Before starting
to write code, the programmer can jot down certain calculations in
a mathematical notation (see Fig.\ \ref{fig:Example-calculation-in-type-theory}).
The results of those calculations will help design the code fragment
the programmer is about to write. This activity is similar to that
of an engineer who performs some mathematical calculations before
embarking on a design project. \begin{wrapfigure}{L}{0.5\textwidth}%
\begin{centering}
{\footnotesize{}\vspace{0.25\baselineskip}
\includegraphics[width=1\linewidth]{ftt-example}\vspace{-0.25\baselineskip}
}{\footnotesize\par}
\par\end{centering}
{\footnotesize{}\caption{A programmer performs a derivation before writing Haskell code.\label{fig:Example-calculation-in-type-theory}}
}{\footnotesize\par}

\vspace{-0.5\baselineskip}
\end{wrapfigure}%
 

\noindent A recent example of a development in applied functional
type theory is the \textsf{``}free applicative functor\textsf{''} construction. It
was first described in a 2014 paper;\footnote{\texttt{\href{https://arxiv.org/pdf/1403.0749.pdf}{https://arxiv.org/pdf/1403.0749.pdf}}}
a couple of years later, a combined free applicative / free monad
data type was designed and its implementation proposed in Scala\footnote{\texttt{\href{https://github.com/typelevel/cats/issues/983}{https://github.com/typelevel/cats/issues/983}}}
as well as in Haskell.\footnote{\texttt{\href{https://elvishjerricco.github.io/2016/04/08/applicative-effects-in-free-monads.html}{https://elvishjerricco.github.io/2016/04/08/applicative-effects-in-free-monads.html}}}
This technique allows programmers to implement declarative side-effect
computations where some parts are sequential but other parts are computed
in parallel, and to achieve the parallelism \emph{automatically} while
maintaining the composability of the resulting programs. The new technique
has advantages over monad transformers, which was a previously known
method of composing declarative side-effects. The combined \textsf{``}free
applicative / free monad\textsf{''} code was designed and implemented by true
software engineers. They first derived the type constructor that has
the necessary algebraic properties. Guided by the resulting type formula,
they wrote code that was guaranteed to work as intended.

Another example of using applied functional type theory is the  \textsf{``}\index{tagless final}tagless
final\textsf{''} encoding of effects, first described\footnote{\texttt{\href{http://okmij.org/ftp/tagless-final/index.html}{http://okmij.org/ftp/tagless-final/index.html}}}
in 2009. That technique (called \textsf{``}Church-encoded free monad\index{free monad}\textsf{''}
in the present book) has advantages over the ordinary free monad in
a number of cases \textemdash{} just as the free monad itself was
used to cure certain problems with monad transformers.\footnote{\texttt{\href{http://blog.ezyang.com/2013/09/if-youre-using-lift-youre-doing-it-wrong-probably/}{http://blog.ezyang.com/2013/09/if-youre-using-lift-youre-doing-it-wrong-probably/}}}
The new encoding is not tied to a specific programming language. Rather,
it is a language-agnostic construction that was originally described
in OCaml and later used in Haskell and Scala, but can be made to work
even in Java,\footnote{\texttt{\href{https://oleksandrmanzyuk.wordpress.com/2014/06/18/}{https://oleksandrmanzyuk.wordpress.com/2014/06/18/}}}
which is not an FP language.

This example shows that we may need several more years of work before
the practical aspects of using applied functional type theory are
sufficiently well understood by the FP community. The theory is in
active development, and its design patterns \textemdash{} as well
as the exact scope of the requisite theoretical material \textemdash{}
are still being figured out. If the 40-year gap hypothesis\footnote{\texttt{\href{https://www.linkedin.com/pulse/40-year-gap-what-has-academic-computer-science-ever-done-winitzki}{https://www.linkedin.com/pulse/40-year-gap-what-has-academic-computer-science-ever-done-winitzki}}}
holds, we should expect applied functional type theory (perhaps under
a different name) to become mainstream by 2030. This book is a step
towards a clear designation of the scope of that theory.

\section{Does software need engineers, or are artisans sufficient? }

The demand for programmers is growing. \textsf{``}Software developer\textsf{''} was
\#1 best job\footnote{\texttt{\href{http://money.usnews.com/money/careers/articles/how-us-news-ranks-the-best-jobs}{http://money.usnews.com/money/careers/articles/how-us-news-ranks-the-best-jobs}}}
in the US in 2018. But is there a demand for engineers or just for
artisans?

We do not seem to be able to train enough software artisans.\footnote{\texttt{\href{http://archive.is/137b8}{http://archive.is/137b8}}}
So, it is probably impossible to train as many software engineers
in the true sense of the word. Modern computer science courses do
not actually train engineers in that sense. Instead, they train academic
researchers who can also work as software artisans and write code.

Looking at the situation in construction business in the U.S.A., we
find that it employs about $10$ times more construction workers as
architects. We might conclude that perhaps one software engineer is
required per dozen software artisans.

What is the price of \emph{not} having engineers, of replacing them
with artisans?

Software practitioners have long bemoaned the permanent state of \textsf{``}crisis\textsf{''}
in software development. Code \textsf{``}rots with time\textsf{''}, its complexity
grows \textsf{``}out of control\textsf{''}, and operating systems have been notorious
for ever-appearing security flaws\footnote{\texttt{\href{http://archive.fo/HtQzw}{http://archive.fo/HtQzw}}}
despite many thousands of programmers and testers employed. It appears
that the growing complexity of software tends to overwhelm the capacity
of the human brain for correct \emph{artisanal} programming.

It is precisely in designing large and robust software systems that
we would benefit from true engineering. Artisans has been building
bridges and using chemical reactions by trial and error and via tradition,
long before mechanical or chemical engineering disciplines were developed
and founded upon rigorous theory. But once the theory became available,
engineers were able to design unimaginably more powerful and complicated
structures, devices, and processes. So, we may expect that trial,
error, and adherence to tradition is inadequate for some of the more
complex software development tasks in front of us. 

To build large and reliable software, such as new mobile or embedded
operating systems or distributed peer-to-peer trust architectures,
we will most likely need the qualitative increase in productivity
and reliability that can only come from replacing artisanal programming
by a true engineering discipline. Applied functional type theory and
functional programming are steps in that direction.

\chapter{Essay: Towards functional data engineering with Scala}

Data engineering is among the highest-demand\footnote{\texttt{\href{http://archive.is/mK59h}{http://archive.is/mK59h}}}
novel occupations in the IT world today. Data engineers create software
pipelines that process large volumes of data efficiently. Why did
the Scala programming language emerge as a premier tool\footnote{\texttt{\href{https://www.slideshare.net/noootsab/scala-the-unpredicted-lingua-franca-for-data-science}{https://www.slideshare.net/noootsab/scala-the-unpredicted-lingua-franca-for-data-science}}}
for crafting the foundational data engineering technologies such as
Spark or Akka? Why is Scala in high demand\footnote{\texttt{\href{https://techcrunch.com/2016/06/14/scala-is-the-new-golden-child/}{https://techcrunch.com/2016/06/14/scala-is-the-new-golden-child/}}}
within the world of big data?

There are reasons to believe that the choice of Scala was not accidental.

\section{Data is math}

Humanity has been working with data at least since Babylonian tax
tables\footnote{\texttt{\href{https://www.nytimes.com/2017/08/29/science/trigonometry-babylonian-tablet.html}{https://www.nytimes.com/2017/08/29/science/trigonometry-babylonian-tablet.html}}}
and the ancient Chinese number books.\footnote{\texttt{\href{https://web.archive.org/web/20170425233550/https://quatr.us/china/science/chinamath.htm}{https://quatr.us/china/science/chinamath.htm}}}
Mathematics summarizes several millennia\textsf{'}s worth of data processing
experience in a few fundamental tenets:

\begin{wrapfigure}{I}{0.34\columnwidth}%
\begin{centering}
\vspace{-0.65\baselineskip}
\includegraphics[width=0.96\linewidth]{type-error}\vspace{-0.5\baselineskip}
\par\end{centering}
\caption{\index{jokes}Mixing incompatible data types produces nonsensical
results.\label{fig:A-nonsensical-calculation}}

\vspace{-3.5\baselineskip}
\end{wrapfigure}%

\begin{itemize}
\item Data is \emph{immutable}, because facts are immutable. 
\item Each \emph{type} of values (population count, land area, distance,
price, location, time, etc.) needs to be handled separately; it is
meaningless to add a distance to a population count.
\item Data processing should be performed according to \emph{mathematical
formulas}. 
\end{itemize}
Violating these tenets produces nonsense (see Fig.\ \ref{fig:A-nonsensical-calculation}
for a real-life illustration).

The power of the principles of mathematics extends over all epochs
and all cultures; math is the same in San Francisco, in Rio de Janeiro,
in Kuala-Lumpur, and in Pyongyang (Fig.\ \ref{fig:The-Pyongyang-method-of-error-free-programming}).

\section{Functional programming is math}

The functional programming paradigm is based on mathematical principles:
values are immutable, data processing is coded through formula-like
expressions, and each type of data is required to match correctly
during the computations. The type-checking process automatically prevents
programmers from making many kinds of coding errors. In addition,
programming languages such as Scala and Haskell have a set of features
adapted to building powerful abstractions and domain-specific languages.
This power of abstraction is not accidental. Since mathematics is
the ultimate art of building abstractions, math-based functional programming
languages capitalize on having several millennia of mathematical experience.

A prominent example of how mathematics informs the design of programming
languages is the connection between constructive logic\footnote{\texttt{\href{https://en.wikipedia.org/wiki/Intuitionistic_logic}{https://en.wikipedia.org/wiki/Intuitionistic\_logic}}}
and the programming language\textsf{'}s type system, called the Curry-Howard
(CH) correspondence. The main idea of the CH correspondence\index{Curry-Howard correspondence}
is to think of programs as mathematical formulas that compute a value
of a certain type $A$. The CH correspondence is between programs
and logical propositions: To any program that computes a value of
type $A$, there corresponds a proposition stating that \textsf{``}a value
of type $A$ can be computed\textsf{''}.

This may sound rather theoretical so far. To see the real value of
the CH correspondence, recall that formal logic has operations \textsf{``}\textbf{\emph{and}}\textsf{''},
\textsf{``}\textbf{\emph{or}}\textsf{''}, and \textsf{``}\textbf{\emph{implies}}\textsf{''}. For any
two propositions $A$, $B$, we can construct the propositions \textsf{``}$A$
\textbf{\emph{and}} $B$\textsf{''}, \textsf{``}$A$ \textbf{\emph{or}} $B$\textsf{''}, \textsf{``}$A$
\textbf{\emph{implies}} $B$\textsf{''}. These three logical operations are
foundational; without one of them, the logic is \emph{incomplete}
(you cannot derive some theorems).

A programming language \textbf{obeys the CH correspondence}\index{Curry-Howard correspondence}
with the logic if for any types $A$, $B$, the language also contains
composite types corresponding to the logical formulas \textsf{``}$A$ \textbf{\emph{or}}
$B$\textsf{''}, \textsf{``}$A$ \textbf{\emph{and}} $B$\textsf{''}, \textsf{``}$A$ \textbf{\emph{implies}}
$B$\textsf{''}. In Scala, these composite types are \lstinline!Either[A, B]!,
the tuple \lstinline!(A,B)!, and the function type \lstinline!A => B!.
All modern functional languages such as OCaml, Haskell, Scala, F\#,
Swift, Elm, and PureScript support these three type constructions
and thus obey the CH correspondence. Having a \emph{complete} logic
in a language\textsf{'}s type system enables declarative domain-driven code
design.\footnote{\texttt{\href{https://fsharpforfunandprofit.com/ddd/}{https://fsharpforfunandprofit.com/ddd/}}}

\begin{wrapfigure}{I}{0.5\columnwidth}%
\begin{centering}
\vspace{-0.5\baselineskip}
\includegraphics[width=1\linewidth]{no-bugs}\vspace{-0.5\baselineskip}
\par\end{centering}
\caption{\index{jokes}The Pyongyang method of error-free software engineering.\label{fig:The-Pyongyang-method-of-error-free-programming}}
\vspace{-3\baselineskip}
\end{wrapfigure}%

It is interesting to note that most older programming languages (C/C++,
Java, JavaScript, Python) do not support some of these composite types.
In other words, these programming languages have type systems based
on an incomplete logic. As a result, users of these languages have
to implement burdensome workarounds that make for error-prone code.
Failure to follow mathematical principles has real costs (Figure~\ref{fig:The-Pyongyang-method-of-error-free-programming}).

\section{The power of abstraction}

Early adopters of Scala, such as Netflix, LinkedIn, and Twitter, were
implementing what is now called \textsf{``}big data engineering\textsf{''}. The required
software needs to be highly concurrent, distributed, and resilient
to failure. Those software companies used Scala as their main implementation
language and reaped the benefits of functional programming.

What makes Scala suitable for big data tasks? The only reliable way
of managing massively concurrent code is to use sufficiently high-level
abstractions that make application code declarative. The two most
important such abstractions are the \textsf{``}resilient distributed dataset\textsf{''}
(RDD) of Apache Spark and the \textsf{``}reactive stream\textsf{''} used in systems
such as Kafka, Akka Streams, and Apache Flink. While these abstractions
are certainly implementable in Java or Python, a fully declarative
and type-safe usage is possible only in a programming language with
a sophisticated type system. Among the currently available mature
functional languages, only Scala and Haskell are technically adequate
for that task, due to their support for typeclasses and higher-order
types. The early adopters of Scala were able to benefit from the powerful
abstractions Scala supports. In this way, Scala enabled those businesses
to engineer and to scale up their massively concurrent computations.

It remains to see why Scala (and not, say, OCaml or Haskell) became
the \emph{lingua franca} of big data.

\section{Scala is Java on math }

The recently invented general-purpose functional programming languages
may be divided into \textsf{``}academic\textsf{''} (OCaml, Haskell) and \textsf{``}industrial\textsf{''}
(F\#, Scala, Swift).

The \textsf{``}academic\textsf{''} languages are clean-room implementations of well-researched
mathematical principles of programming language design (the CH correspondence
being one such principle). These languages are not limited by requirements
of compatibility with any existing platforms or libraries. Because
of this, the \textsf{``}academic\textsf{''} languages have been designed and used
for pursuing various mathematical ideas to their logical conclusion.\footnote{OCaml has arbitrary recursive product and co-product types that can
be freely combined with object-oriented types. Haskell removes all
side effects from the language and supports partial type functions
of arbitrarily high order.} At the same time, software practitioners struggle to adopt these
programming languages due to a steep learning curve, a lack of enterprise-grade
libraries and tool support, and immature package management.

The languages from the \textsf{``}industrial\textsf{''} group are based on existing
and mature software ecosystems: F\# on .NET, Scala on JVM, and Swift
on the MacOS/iOS platform. One of the important design requirements
for these languages is 100\% binary compatibility with their \textsf{``}parent\textsf{''}
platforms and languages (F\# with C\#, Scala with Java, and Swift
with Objective-C). Because of this, developers can immediately take
advantage of the existing tooling, package management, and industry-strength
libraries, while slowly ramping up the idiomatic usage of new language
features. However, the same compatibility requirements dictate certain
limitations in the languages, making their design less than fully
satisfactory from the functional programming viewpoint.

It is now easy to see why the adoption rate of the \textsf{``}industrial\textsf{''}
group of languages is much higher\footnote{\texttt{\href{https://www.tiobe.com/tiobe-index/}{https://www.tiobe.com/tiobe-index/}},
archived in 2019 at \texttt{\href{http://archive.is/RsNH8}{http://archive.is/RsNH8}}} than that of the \textsf{``}academic\textsf{''} languages. The transition to the
functional paradigm is also smoother for software developers because
F\#, Scala, and Swift seamlessly support the familiar object-oriented
programming\index{object-oriented programming} paradigm. At the same
time, these new languages still have logically complete type systems,
which gives developers an important benefit of type-safe domain modeling.

Nevertheless, the type systems of these languages are not equally
powerful. For instance, F\# and Swift are similar to OCaml in many
ways but omit OCaml\textsf{'}s parameterized modules and some other features.
Of all mentioned languages, only Scala and Haskell directly support
typeclasses and higher-order functions on types, which are helpful
for expressing abstractions such as automatically parallelized data
sets or asynchronous data streams.

To see the impact of these advanced features, consider LINQ, a domain-specific
language for database queries on .NET, implemented in C\# and F\#
through a special built-in syntax supported by Microsoft\textsf{'}s compilers.
Analogous functionality is provided in Scala as a \emph{library},
without need to modify the Scala compiler, by several open-source
projects such as Slick and Quill. Similar libraries exist for Haskell
\textemdash{} but not in languages with less powerful type systems.

\section{Summary}

Only Scala has all of the features required for industrial-grade functional
programming:
\begin{enumerate}
\item Functional collections in the standard library.
\item A sophisticated type system with support for typeclasses and higher-order
types.
\item Seamless compatibility with a mature software ecosystem (JVM).
\end{enumerate}
Based on this assessment, we may be confident in Scala\textsf{'}s future as
a main implementation language for big data engineering.

\chapter{Essay: Why category theory is useful in functional programming}

\index{category theory!in functional programming}This essay is for
readers who are already somewhat familiar with category theory.

\section{A \textquotedblleft types/functions\textquotedblright{} category for
a programming language}

We consider programming languages that support various data types,
such as integers (\lstinline!Int!), floating-point numbers (\lstinline!Float!),
strings (\lstinline!String!), arrays of strings (\lstinline!Array[String]!),
and so on. Such languages allow programmers to define functions with
specified types of arguments and return values. The compiler will
then verify that all functions are always applied to arguments of
correct types, and also that all return values have the expected types. 

To each programming language of that kind, there corresponds a \textsf{``}types/functions\textsf{''}
category:
\begin{itemize}
\item The objects of the category are all the data types supported by the
language (including user-defined data types). As an example, for Scala
there will be an object \lstinline!Int!, an object \lstinline!Float!,
an object \lstinline!String!, an object \lstinline!Array[String]!,
and so on.
\item The morphisms between objects \lstinline!A! and \lstinline!B! are
all functions implementable in the language (with finitely long program
code) that take a single argument of type \lstinline!A! and return
a value of type \lstinline!B!.
\item We assume that the computer has countably infinite memory, so objects
can be viewed as (at most) countably infinite sets. Morphisms will
also form at most countably infinite sets.
\end{itemize}
The category defined in this way will typically have a large number
of morphisms between most objects. For example, morphisms between
objects \lstinline!Boolean! and \lstinline!Int! are functions that
take a single argument of type \lstinline!Boolean! and return a value
of type \lstinline!Int!. There are as many such functions as pairs
of integers. Scala code for one of those morphisms looks like this:
\begin{lstlisting}
def morphismBooleanToInt: Boolean => Int = { b => if (b) 123 else 456 }
\end{lstlisting}

Why do the category laws hold? The composition of morphisms corresponds
to composition of functions, which we can implement by writing code
that applies the first function and then applies the second function.
In Scala:
\begin{lstlisting}
def composeMorphisms[A, B, C](f: A => B, g: B => C): A => C   =   { a => g(f(a)) }
\end{lstlisting}
Equivalent functionality can be implemented in most programming languages.

The category\textsf{'}s identity law says that there must be a morphism between
objects \lstinline!A! and \lstinline!A!. This can be implemented
in most programming languages as a function that returns its argument
unchanged:
\begin{lstlisting}
def identity[A]: A => A = { x => x }
\end{lstlisting}
One can check that morphism composition is associative and agrees
with the identity morphism.

For a given programming language, we have thus defined the \textsf{``}types/functions
category\textsf{''}, which can be seen as a subcategory of the category of
sets. Most of the time, we will be working with that category, or
with the category of its endofunctors, or with a sub-category of these
categories.

Different programming languages will give rise to different \textsf{``}types/functions\textsf{''}
categories, but all those categories have many common features that
are especially important in languages designed for functional programming
(the \textsf{``}FP languages\textsf{''}, such as OCaml, Haskell, Scala and others).

\section{The use of endofunctors}

An endofunctor in the \textsf{``}types/functions\textsf{''} category is a mapping
of types together with a mapping of functions. A good example is the
\lstinline!Array! data type. In some programming languages, the type
of an array\textsf{'}s elements can be specified and enforced throughout the
code. For example, in Scala one can use the type \lstinline!Array[Int]!
for an array of integers, \lstinline!Array[String]! for an array
of strings, \lstinline!Array[Array[Int]]! for an array containing
nested arrays of integers, etc. So, \lstinline!Array! can be seen
as a mapping from types to types: it maps the type \lstinline!Int!
to the type \lstinline!Array[Int]!, the type \lstinline!String!
to the type \lstinline!Array[String]!, etc. For any type \lstinline!X!,
we have the type \lstinline!Array[X]!. This is the object-to-object
map of an endofunctor.

An endofunctor also needs a map from morphisms to morphisms. Given
a function \lstinline!f: X => Y!, we need to implement a function
of type \lstinline!Array[X] => Array[Y]!. This can be done by writing
a loop over the array and applying the function \lstinline!f! to
each element (of type \lstinline!X!). The resulting values (of type
\lstinline!Y!) are then collected in a new array, of type \lstinline!Array[Y]!.

This code can be written in many programming languages in a generic
manner, using type parameters such as \lstinline!X! and \lstinline!Y!.
The same code will then work for arrays and functions of any given
type. In Scala, the code could be written as the following function
(usually called \lstinline!fmap! in FP libraries):
\begin{lstlisting}
def fmap[X, Y: ClassTag](f: X => Y): Array[X] => Array[Y] = { arrayX: Array[X] =>
  val arrayY = new Array[Y](arrayX.size)
  for { i <- arrayX.indices } arrayY(i) = f(arrayX(i))
  arrayY  // Return this array of type Array[Y].
}
\end{lstlisting}
One can then check that the code of \lstinline!fmap! satisfies the
identity and composition laws of endofunctors. This completes the
implementation of the \lstinline!Array! endofunctor.

Why does \lstinline!fmap! satisfy the laws of endofunctors? The categorical
properties of functions are preserved if we apply functions to each
element of an array and collect the results \emph{in the same order}.
An identity function applied to every element will not modify the
array. Function composition is preserved because a composition of
two functions will be applied separately to each array element.

The same construction can be applied to many data structures other
than arrays. It turns out that many programs can be reformulated using
the operation of applying a function to every value in a data structure
(i.e., the function \lstinline!fmap!). This reformulation leads to
code that avoids loops: the loops are replaced by \lstinline!fmap!
functions of some endofunctors, and all those \lstinline!fmap! functions
are implemented in a standard library. In practice, code written via
\lstinline!fmap! instead of loops is more concise and admits fewer
opportunities for errors. The programmer\textsf{'}s intuition about \textsf{``}applying
functions to every value held within a data structure\textsf{''} is then directly
represented by the formal laws of endofunctors. Once those laws are
verified, the programmer is assured that the code written via \lstinline!fmap!
will work according to the programmer\textsf{'}s intuitive expectations.

\section{The use of natural transformations}

What is a natural transformation between endofunctors in the \textsf{``}types/functions\textsf{''}
category? For two given endofunctors \lstinline!F! and \lstinline!G!,
a natural transformation \lstinline!t: F ~> G! is defined by its
components. The component at object \lstinline!X! is a function of
type \lstinline!F[X] => G[X]!; this must be defined for all \lstinline!X!.
Some programming languages support functions with type parameters.
In Scala, the syntax is
\begin{lstlisting}
def t[X]: F[X] => G[X] = ...
\end{lstlisting}
The code of such a function is written once and will work in the same
way for all types \lstinline!X!.

An example of natural transformation is a function that reverses the
order of elements in an array:
\begin{lstlisting}
def reverse[X]: Array[X] => Array[X] = ...
\end{lstlisting}
The algorithm is \textsf{``}fully parametric\textsf{''}: it is written in the same
way for all type parameters \lstinline!X!.

It turns out that, by the Reynolds-Wadler parametricity theorem, any
code written in a fully parametric manner will satisfy the law of
a natural transformation (the naturality law). The naturality law
states that applying the endofunctor \lstinline!F!\textsf{'}s morphism map
before a natural transformation \lstinline!t! must be equal to applying
the endofunctor \lstinline!G!\textsf{'}s map after \lstinline!t!. In Scala
syntax, the law is written as
\begin{lstlisting}
t(fmap_F(f)(x)) == fmap_G(f)(t(x))
\end{lstlisting}
This law can be verified directly for a given code of \lstinline!t!
and with known code for \lstinline!fmap_F! and \lstinline!fmap_G!.

Naturality laws are satisfied by transformations that rearrange data
items in a data structure in some way that does not depend on specific
values or types. In this way, the formal laws of natural transformations
directly represent programmers\textsf{'} intuitions about code that works \textsf{``}in
the same way for all type parameters\textsf{''}.

As we have just seen, the notions of endofunctors and natural transformations
are useful in programming languages that support types with type parameters
(such as \lstinline!Array[X]!) and functions with type parameters
(such as \lstinline!reverse[X]!). Programming languages that do not
support those features cannot benefit from the powerful reasoning
tools of category theory.

\section{Other properties of the \textquotedblleft types/functions\textquotedblright{}
category}

Morphisms in the \textsf{``}types/functions\textsf{''} category are always functions
of a single argument. However, programming languages usually support
functions with many arguments. There are two ways to imitate such
functions: tupling and currying.

Tupling means that we put all arguments into a compound structure
(a pair, a triple, etc.). The function is still viewed as having a
single argument, but the type of that argument is the type of a pair,
or a triple, or a longer tuple type. This works when the programming
language supports tuple types. A tupled function\textsf{'}s type is written
(in Scala) as, e.g., \lstinline!((A, B, C)) => D!.

Tuple types correspond to finite products of objects in the \textsf{``}types/functions\textsf{''}
category. So, it is useful if the category has (finite) products. 

Currying means that we create a function that takes the first argument
and returns a curried function that handles the rest of the arguments
in the same way (takes the second argument and again returns a function,
etc.). A curried function\textsf{'}s type is written in Scala as, e.g., \lstinline!A => B => C => D!.
To support this method, the programming language should have function
types. The corresponding categorical construction is the \textsf{``}exponential\textsf{''}
object. 

In the practice of functional programming, it has been found useful
to have the type \lstinline!Unit!, which has exactly one value, and
the type \lstinline!Nothing!, which has no values. In the \textsf{``}types/functions\textsf{''}
category, these types correspond to the terminal and the initial objects.

Finally, disjunctive types correspond to co-products in the \textsf{``}types/functions\textsf{''}
category.

In this way, we find that various well-known mathematical properties
of the \textsf{``}types/functions\textsf{''} category (initial and terminal objects,
finite products and co-products, exponentials) correspond to properties
of the programming language that proved useful in the practice of
software engineering.

\section{Some useful sub-categories of endofunctors}

Besides loops that apply functions to array elements, other frequently
used computations are nested loops and \textsf{``}while\textsf{''}-loops that are
repeated while a given condition holds and then stopped. It turns
out that category theory provides a convenient language for reasoning
about such computations. Similarly to representing loops via endofunctors,
the various kinds of loops are encoded via certain sub-categories
of endofunctors in the \textsf{``}types/functions\textsf{''} category.

To see how this works, we need to define an auxiliary sub-category
called the \textsf{``}\lstinline!F!-lifted\textsf{''} (where \lstinline!F! may be
any given endofunctor, such as \lstinline!Array!). The \lstinline!F!-lifted
sub-category is the image of the endofunctor \lstinline!F!. The objects
of that sub-category are types of the form \lstinline!F[A]!. The
morphisms of that sub-category are functions of type \lstinline!F[A] => F[B]!
(not necessarily obtained by lifting some function \lstinline!f: A => B!
through the endofunctor \lstinline!F!\textsf{'}s \lstinline!fmap! method). 

\subsection{Filterable endofunctors}

To describe \textsf{``}while\textsf{''}-loops using category theory, we begin by reformulating
the loop as a \emph{mathematical function} rather than as a sequence
of computer instructions. To be specific, consider a program with
a loop that stops when a certain condition first becomes \lstinline!false!.
A loop of this kind may be modeled by a function that takes an initial
array as argument and returns a new array that is truncated when a
given condition first becomes \lstinline!false!. The condition is
represented by a function evaluated on each element of the array.
The Scala standard library includes such a function (\lstinline!takeWhile!).
An example of its usage is:
\begin{lstlisting}[mathescape=true]
scala> Array(1, 2, 3, 4, 5).takeWhile(x => x < 4)    // Truncate the array when $\color{dkgreen} x \geq 4$ is first found.
res0: Array[Int] = Array(1, 2, 3) 
\end{lstlisting}
The next step is to extend this function to work with arbitrary types
instead of integers. The type signature of \lstinline!takeWhile!
may be written as
\begin{lstlisting}
def takeWhile[X](p: X => Boolean): Array[X] => Array[X]
\end{lstlisting}
Here \lstinline!X! is a type parameter. The first argument is a predicate
of type \lstinline!X => Boolean!.

Finally, we write the laws that we expect this function to satisfy.
For instance, if the given predicate \lstinline!p! always returns
\lstinline!true!, the function should not change the given array
(the \textsf{``}identity law\textsf{''}):
\begin{lstlisting}
takeWhile(x => true)  ==  identity
\end{lstlisting}
Another plausible law is called the \textsf{``}composition law\textsf{''}. If we apply
\lstinline!takeWhile! with a predicate \lstinline!p1! and then again
apply \lstinline!takeWhile! with another predicate \lstinline!p2!,
the resulting truncated array should be the same as if we applied
\lstinline!takeWhile! just once with a Boolean conjunction of \lstinline!p1!
and \lstinline!p2!: 
\begin{lstlisting}
takeWhile(p1) andThen takeWhile(p2) == takeWhile( x => p1(x) && p2(x) )
\end{lstlisting}

The identity and composition laws of \lstinline!takeWhile! are analogous
to the identity and composition laws of functors. More precisely,
one can derive the laws of \lstinline!takeWhile! from the laws of
an auxiliary functor between a certain Kleisli category and the \lstinline!F!-lifted
category (see Example~\ref{subsec:Example-category-definition-of-filterable-functor}
for details). That auxiliary functor exists only for some endofunctors
\lstinline!F!, which are called \textsf{``}filterable\textsf{''} in this book. Filterable
endofunctors are a sub-category of all endofunctors of the \textsf{``}types/functions\textsf{''}
category. 

With this construction, one may now regard the laws of \lstinline!takeWhile!
not as arbitrarily postulated properties but as a consequence of the
functor laws. In this way, category theory validates the programmer\textsf{'}s
intuition for the choice of the laws for \lstinline!takeWhile!.

\subsection{Monadic endofunctors}

To evaluate a nested loop, which may be written in Scala as, e.g.,
\begin{lstlisting}
for {
  x <- 1 to 10
  y <- 1 to x / 2
} yield f(x, y) // Some computation that may use x and y.
\end{lstlisting}
the computer will perform ten repetitions of the inner loop over \lstinline!y!.
This computation is equivalent to converting the nested loop into
an ordinary, \textsf{``}flattened\textsf{''} loop that has a larger total number of
repetitions (in this example, $25$ repetitions). To describe this
situation using category theory, we start by reformulating a nested
loop into a mathematical function. The arguments of that function
are the first array (\lstinline!1 to 10!) for iterating with \lstinline!x!,
and a function from a value of \lstinline!x! to the nested array
(\lstinline!x => 1 to x / 2!). The function returns a \textsf{``}flattened\textsf{''}
array of $25$ values. 

The Scala library contains such a function, named \lstinline!flatMap!.
An example of usage is:
\begin{lstlisting}
scala> (1 to 10).toArray.flatMap(x => 1 to x / 2)
res0: Array[Int] = Array(1, 1, 1, 2, 1, 2, 1, 2, 3, 1, 2, 3, 1, 2, 3, 4, 1, 2, 3, 4, 1, 2, 3, 4, 5)
\end{lstlisting}
This function can be used repeatedly to convert arbitrarily deeply
nested loops into \textsf{``}flat\textsf{''} loops.

The next step is to formulate a fully parametric type signature for
\lstinline!flatMap!:
\begin{lstlisting}
def flatMap[A, B](f: A => Array[B]): Array[A] => Array[B]
\end{lstlisting}
In this way, \lstinline!flatMap! can transform arrays with elements
of any type.

The \lstinline!flatMap! function must satisfy certain properties
that are useful for practical programming. One of these properties
is \textsf{``}associativity\textsf{''}. A deeply nested loop may be flattened by applying
\lstinline!flatMap! first to the outer layers and then to the inner
layers, or by applying flatMap first to the inner layers; the results
must be the same. This and other properties of \lstinline!flatMap!
are analogous to the laws of a category: there are two identity laws
and one associativity law. More precisely, one can derive the laws
of \lstinline!flatMap! from the requirement that the Kleisli category
on the endofunctor \lstinline!Array! is well-defined (see Section~\ref{subsec:Monads-in-category-theory-monad-morphisms}
for details). This is equivalent to saying that \lstinline!Array!
is a monad. Monads form a sub-category of endofunctors of the \textsf{``}types/functions\textsf{''}
category.

\section{Category theory and the laws of FP idioms}

We have seen that \textsf{``}while\textsf{''}-loops and nested loops can be reformulated
through type-parameterized functions satisfying certain laws. Those
laws are then equivalent to the laws of suitably chosen functors or
categories. This turns out to be a general pattern:
\begin{itemize}
\item Begin with a known idiom of computation (e.g., a certain kind of a
loop).
\item Reformulate that idiom through functions with parameterized argument
types.
\item Write the laws that programmers expect those functions to satisfy.
\item Prove that those laws are equivalent to the laws of a suitable functor
and/or category.
\end{itemize}
The derivations in Chapters~\ref{chap:Functors,-contrafunctors,-and}\textendash \ref{chap:monad-transformers}
of this book follow this pattern. One can show that this pattern holds
for at least \emph{eleven} sub-categories of endofunctors used in
FP practice: functors, contrafunctors, filterable functors, filterable
contrafunctors, applicative functors, applicative contrafunctors,
monads, comonads, traversable functors, monad transformers, and comonad
transformers. 

It appears that category theory and its basic tools (functors, natural
transformations, commutative diagrams) provide a powerful and versatile
language for reasoning about laws of various FP idioms. By invoking
category theory, programmers avoid having to memorize a large number
of laws and constructions. Without the underlying categorical justification,
the laws for different endofunctors will appear to be chosen arbitrarily,
with no clearly recognizable system or pattern.

In addition, category theory guides programmers in creating highly
abstract libraries that work uniformly with all endofunctors of a
certain sub-category. In programmer\textsf{'}s terms, such libraries contain
functions parameterized by type constructors satisfying appropriate
constraints. Examples are functions that define the product or the
co-product of any two given functors, or define a free monad on a
given functor. Implementing libraries of that kind requires formulating
and verifying the relevant laws. Category theory is a reliable foundation
for such libraries.
